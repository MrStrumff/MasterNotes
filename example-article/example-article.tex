\documentclass[ikea]{../ceri/sty/MasterArticle}

\title{Expression des $\zeta(2k)$}
\author{Jean \textsc{Martin}}
\date{}

\newcommand{\e}{\mathrm{e}}
\renewcommand{\i}{\mathrm{i}}
\newcommand{\U}{\mathbf{U}}
\newcommand{\M}{\mathscr{M}}
\newcommand{\defeq}{\overset{\text{def}}=}
\renewcommand{\O}{\mathbf{O}}
\newcommand{\F}{\mathbf{F}}
\renewcommand{\S}{\mathfrak{S}}
\newcommand{\A}{\mathfrak{A}}
\newcommand{\GL}{\mathbf{GL}}
\newcommand{\Id}{\mathrm{Id}}
\renewcommand{\P}{\mathbb{P}}
\renewcommand{\emptyset}{\varnothing}
\newcommand{\inter}{\mathring}

\let\Re\relax
\let\Im\relax
\DeclareMathOperator{\Re}{\mathfrak{Re}}
\DeclareMathOperator{\Im}{\mathfrak{Im}}

\let\ker\relax
\DeclareMathOperator{\ker}{Ker}
\DeclareMathOperator{\im}{Im}

\DeclareMathOperator\rg{rg}
\DeclareMathOperator\sgn{sgn}
\DeclareMathOperator\Rg{Rg}
\DeclareMathOperator\tr{tr}
\DeclareMathOperator\Tr{Tr}
\DeclareMathOperator\Hess{Hess}
\DeclareMathOperator\supp{supp}
\DeclareMathOperator\Stab{Stab}
\DeclareMathOperator\Sp{Sp}
\DeclareMathOperator\Card{Card}
\DeclareMathOperator\diag{diag}
\DeclareMathOperator\Mat{Mat}
\DeclareMathOperator\Det{Det}
\DeclareMathOperator\Is{Is}
\DeclareMathOperator\Conv{Conv}
\DeclareMathOperator\Vect{Vect}
\DeclareMathOperator\dist{dist}
\DeclareMathOperator\pgcd{pgcd}
\DeclareMathOperator\ord{ord}
\DeclareMathOperator*\argmin{arg\;min}

\newcommand{\norme}[1]{\left\lVert #1\right\rVert}
\newcommand{\abs}[1]{\left\lvert #1\right\rvert}

\newcommand{\normet}[1]{{\left\vert\kern-0.25ex\left\vert\kern-0.25ex\left\vert #1
            \right\vert\kern-0.25ex\right\vert\kern-0.25ex\right\vert}}

\newcommand{\ps}[2]{\left\langle #1,\,#2\right\rangle}

\newcommand{\transp}[1]{{}^{\mathsf{t}}{#1}}

\renewcommand{\d}{\mathop{}\mathopen{}\mathrm d}
\renewcommand{\bar}{\overline}
\newcommand\restr[1]{\mathclose{}|\mathopen{}_{#1}}
\renewcommand{\bf}{\mathbf}

\makeatletter
\renewenvironment{cases}[1][l]{\matrix@check\cases\env@cases{#1}}{\endarray\right.}
\def\env@cases#1{%
\let\@ifnextchar\new@ifnextchar
\left\lbrace\def\arraystretch{1.2}%
\array{@{}#1@{\quad}l@{}}}
\makeatother

\newcommand{\cyc}[1]{\begin{pmatrix}#1\end{pmatrix}}
\newcommand{\eng}[1]{\left\langle#1\right\rangle}

\let\originalleft\left
\let\originalright\right
\renewcommand{\left}{\mathopen{}\mathclose\bgroup\originalleft}
\renewcommand{\right}{\aftergroup\egroup\originalright}

% Intervalles
\renewcommand{\intff}[2]{\left[#1\,;#2\right]}
\renewcommand{\intof}[2]{\left]#1\,;#2\right]}
\renewcommand{\intfo}[2]{\left[#1\,;#2\right[}
\renewcommand{\intoo}[2]{\left]#1\,;#2\right[}

\usepackage{stmaryrd}

% Intervalles d'entiers
\newcommand{\inteff}[2]{\left\llbracket#1\,;#2\right\rrbracket}
\newcommand{\inteof}[2]{\left\rrbracket#1\,;#2\right\rrbracket}
\newcommand{\intefo}[2]{\left\llbracket#1\,;#2\right\llbracket}
\newcommand{\inteoo}[2]{\left\rrbracket#1\,;#2\right\llbracket}

\newcommand{\addfig}[1]{% add a custom TikZ figure (mostly drawn on https://www.mathcha.io or https://www.q.uiver.app)
    \begin{figure*}[h]
        \centering
        \input{figs/#1.tex}
    \end{figure*}
}

\renewcommand{\thefootnote}{(\arabic{footnote})} % Using symbols for footnotes

\newcommand{\fonction}[5]{
    \begin{array}{lccc}
        #1\colon & #2 & \longrightarrow & #3 \\
                 & #4 & \longmapsto     & #5
    \end{array}
}
\newcommand{\fct}[4]{
    \begin{array}{lccc}
        #1 & \longrightarrow & #2 \\
        #3 & \longmapsto     & #4
    \end{array}
}

\usepackage{csquotes}
\usepackage[
    style=alphabetic,
    backend=biber
]{biblatex}
\DeclareFieldFormat{labelalpha}{\mkbibbold{#1}}
\DeclareFieldFormat{extraalpha}{\mkbibbold{\mknumalph{#1}}}
\bibliography{../ceri/main_biblio.bib}
\usepackage{footmisc}

\usepackage{fontawesome}
\newcommand{\lecon}[2]{% Pour l'environnement <recase>
    \item[
                \ifnum#2=1\relax (\faStar\faStarO\faStarO\faStarO\faStarO) \fi
                \ifnum#2=2\relax (\faStar\faStar\faStarO\faStarO\faStarO) \fi
                \ifnum#2=3\relax (\faStar\faStar\faStar\faStarO\faStarO) \fi
                \ifnum#2=4\relax (\faStar\faStar\faStar\faStar\faStarO) \fi
                \ifnum#2=5\relax (\faStar\faStar\faStar\faStar\faStar) \fi
                \textbf{Leçon #1.}
          ]%
    %
    \ifnum#1=101\relax Groupe opérant sur un ensemble. Exemples et applications. \fi
    \ifnum#1=102\relax Groupe des nombres complexes de module $1$. Sous-groupes des racines de l'unité. Applications. \fi
    \ifnum#1=103\relax Conjugaison dans un groupe. Exemples de sous-groupes distingués et de groupes quotients. Applications. \fi
    \ifnum#1=104\relax Groupes abéliens et non abéliens finis. Exemples et applications. \fi
    \ifnum#1=105\relax Groupe des permutations d'un ensemble fini. Applications. \fi
    \ifnum#1=106\relax Groupe linéaire d'un espace vectoriel de dimension finie $E$, sous-groupes de $\GL(E)$. Applications. \fi
    \ifnum#1=108\relax Exemples de parties génératrices d'un groupe. Applications. \fi
    \ifnum#1=120\relax Anneaux $\Z/n\Z$. Applications. \fi
    \ifnum#1=121\relax Nombres premiers. Applications. \fi
    \ifnum#1=122\relax Anneaux principaux. Applications. \fi
    \ifnum#1=123\relax Corps finis. Applications. \fi
    \ifnum#1=125\relax Extensions de corps. Exemples et applications. \fi
    \ifnum#1=126\relax Exemples d'équations en arithmétique. \fi
    \ifnum#1=141\relax Polynômes irréductibles à une indéterminée. Corps de rupture. Exemples et applications. \fi
    \ifnum#1=142\relax PGCD et PPCM, algorithmes de calcul. Applications. \fi
    \ifnum#1=144\relax Racines d'un polynôme. Fonctions symétriques élémentaires. Exemples et applications. \fi
    \ifnum#1=149\relax Valeurs propres, vecteurs propres. Calculs exacts ou approchés d'éléments propres. Applications. \fi
    \ifnum#1=150\relax Exemples d'actions de groupes sur les espaces de matrices. \fi
    \ifnum#1=151\relax Dimension d'un espace vectoriel (on se limitera au cas de la dimension finie). Rang. Exemples et applications. \fi
    \ifnum#1=152\relax Déterminant. Exemples et applications. \fi
    \ifnum#1=153\relax Polynômes d'endomorphisme en dimension finie. Réduction d'un endomorphisme en dimension finie. Applications. \fi
    \ifnum#1=154\relax Sous-espaces stables par un endomorphisme ou une famille d'endomorphismes d'un espace vectoriel de dimension finie. Applications. \fi
    \ifnum#1=155\relax Endomorphismes diagonalisables en dimension finie. \fi
    \ifnum#1=156\relax Exponentielle de matrices. Applications. \fi
    \ifnum#1=157\relax Endomorphismes trigonalisables. Endomorphismes nilpotents. \fi
    \ifnum#1=158\relax Matrices symétriques réelles, matrices hermitiennes. \fi
    \ifnum#1=159\relax Formes linéaires et dualité en dimension finie. Exemples et applications. \fi
    \ifnum#1=160\relax Endomorphismes remarquables d'un espace vectoriel euclidien (de dimension finie). \fi
    \ifnum#1=161\relax Distances et isométries d'un espace affine euclidien. \fi
    \ifnum#1=162\relax Systèmes d'équations linéaires ; opérations élémentaires, aspects algorithmiques et conséquences théoriques. \fi
    \ifnum#1=170\relax Formes quadratiques sur un espace vectoriel de dimension finie. Orthogonalité, isotropie. Applications. \fi
    \ifnum#1=171\relax Formes quadratiques réelles. Coniques. Exemples et applications. \fi
    \ifnum#1=181\relax Barycentres dans un espace affine réel de dimension finie, convexité. Applications. \fi
    \ifnum#1=190\relax Méthodes combinatoires, problèmes de dénombrement. \fi
    \ifnum#1=191\relax Exemples d'utilisation des techniques d'algèbre en géométrie. \fi
    %
    \ifnum#1=201\relax Espaces de fonctions. Exemples et applications. \fi
    \ifnum#1=203\relax Utilisation de la notion de compacité. \fi
    \ifnum#1=204\relax Connexité. Exemples et applications. \fi
    \ifnum#1=205\relax Espaces complets. Exemples et applications. \fi
    \ifnum#1=207\relax Prolongement de fonctions. Exemples et applications. \fi
    \ifnum#1=208\relax Espaces vectoriels normés, applications linéaires continues. Exemples. \fi
    \ifnum#1=209\relax Approximation d'une fonction par des fonctions régulières. Exemples et applications. \fi
    \ifnum#1=213\relax Espaces de \textsc{Hilbert}. Bases hilbertiennes. Exemples et applications. \fi
    \ifnum#1=214\relax Théorème d'inversion locale, théorème des fonctions implicites. Exemples et applications en analyse et en géométrie. \fi
    \ifnum#1=215\relax Applications différentiables définies sur un ouvert de $\R^n$. Exemples et applications. \fi
    \ifnum#1=219\relax Extremums : existence, caractérisation, recherche. Exemples et applications. \fi
    \ifnum#1=220\relax Équations différentielles ordinaires. Exemples de résolution et d'études de solutions en dimension $1$ et $2$. \fi
    \ifnum#1=221\relax Équations différentielles linéaires. Systèmes d'équations différentielles linéaires. Exemples et applications. \fi
    \ifnum#1=222\relax Exemples d'études d'équations différentielles linéaires et d'équations aux dérivées partielles linéaires. \fi
    \ifnum#1=223\relax Suites numériques. Convergence, valeurs d'adhérence. Exemples et applications. \fi
    \ifnum#1=226\relax Suites vectorielles et réelles définies par une relation de récurrence $u_{n+1}=f(u_n)$. Exemples. Applications à la résolution approchée d'équations. \fi
    \ifnum#1=228\relax Continuité, dérivabilité des fonctions réelles d'une variable réelle. Exemples et applications. \fi
    \ifnum#1=229\relax Fonctions monotones. Fonctions convexes. Exemples et applications. \fi
    \ifnum#1=230\relax Séries de nombres réels ou complexes. Comportement des restes ou des sommes partielles des séries numériques. Exemples. \fi
    \ifnum#1=234\relax Fonctions et espaces de fonctions \textsc{Lebesgue}-intégrables. \fi
    \ifnum#1=235\relax Problèmes d'interversion de limites et d'intégrales. \fi
    \ifnum#1=236\relax Illustrer par des exemples quelques méthodes de calcul d'intégrales de fonctions d'une ou plusieurs variables. \fi
    \ifnum#1=239\relax Fonctions définies par une intégrale dépendant d'un paramètre. Exemples et applications. \fi
    \ifnum#1=241\relax Suites et séries de fonctions. Exemples et contre-exemples. \fi
    \ifnum#1=243\relax Séries entières, propriétés de la somme. Exemples et applications. \fi
    \ifnum#1=245\relax Fonctions d'une variable complexe. Exemples et applications. \fi
    \ifnum#1=246\relax Séries de \textsc{Fourier}. Exemples et applications. \fi
    \ifnum#1=250\relax Transformation de \textsc{Fourier}. Applications. \fi
    \ifnum#1=253\relax Utilisation de la notion de convexité en analyse. \fi
    \ifnum#1=261\relax Loi d'une variable aléatoire : caractérisations, exemples, applications. \fi
    \ifnum#1=262\relax Convergences d'une suite de variables aléatoires. Théorèmes limite. Exemples et applications. \fi
    \ifnum#1=264\relax Variables aléatoires discrètes. Exemples et applications. \fi
    \ifnum#1=265\relax Exemples d'études et d'applications de fonctions usuelles et spéciales. \fi
    \ifnum#1=266\relax Illustration de la notion d'indépendance en probabilités. \fi
    \ifnum#1=267\relax Exemples d'utilisation de courbes en dimension 2 ou supérieure. \fi
}

\newenvironment{recase}{\noindent\textbf{Recasages possibles.}\begin{itemize}[wide]}{\end{itemize}
    % fg, fg2, th, defini, rema, coro
    \par\noindent
    {\color{fg}\rule{.15\textwidth}{1pt}}%
    {\color{fg2}\rule{.11\textwidth}{1pt}}%
    {\color{th}\rule{.0833\textwidth}{1pt}}%
    {\color{defini}\rule{.0666\textwidth}{1pt}}%
    {\color{rema}\rule{.0555\textwidth}{1pt}}%
    {\color{coro}\rule{.0555\textwidth}{1pt}}%
    \\
}

\usepackage{tkz-tab} % Pour les tableaux de variations

\begin{document}
\maketitle

On rappelle que l'\textbf{exponentielle complexe} est la fonction entière définie par
\[ \forall z\in\C,\ \e^z\defeq\sum_{n=0}^{+\infty}\frac{z^n}{n!}. \]

\begin{theo*}[{\autocite[305]{book:FGNAn2}}]
    Soit $f\colon z\in\C^*\mapsto\dfrac{z}{\e^z-1}$. Alors $f$ est développable en série entière en $0$, et on a
    \[ \forall z\in D(0,2\pi),\ f(z) = 1 - \frac z2 + 2\sum_{k=1}^{+\infty}\frac{(-1)^{k-1}}{(2\pi)^{2k}}\Big(\sum_{n=1}^{+\infty}\frac 1{n^{2k}}\Big)z^{2k}\text. \]
\end{theo*}

\begin{demo}
    Soit $z\in\C\setminus 2\i\pi\Z$. Pour montrer ce théorème, on va utiliser le développement en série de \textsc{Fourier} de la fonction $\varphi\colon x\in\intoo{-\pi}\pi\mapsto\exp\left( \dfrac{zx}{2\pi} \right)$. On vérifie aisément que c'est une fonction continue par morceaux sur $\R$. On peut donc calculer ses coefficients de \textsc{Fourier}. On calcule, pour $n\in\Z^*$ :
    \begin{align*}
        c_n(\varphi) & = \frac 1{2\pi}\int_{-\pi}^\pi\varphi(x)\e^{-\i nx}\d x = \frac 1{2\pi}\int_{-\pi}^{\pi}\exp\left( \left( \frac z{2\pi}-\i n \right)x \right)\d x                           \\
                     & = \frac 1{2\pi}\frac 1{(z/2\pi)-\i n}\left[ \e^{\frac{zx}{2\pi}}\e^{-\i nx} \right]_{-\pi}^\pi = \frac{(-1)^n\left( \e^{\frac z2}-\e^{-\frac z2} \right)}{z-2\pi\i n}\text.
    \end{align*}
    De plus, $\varphi$ est de classe $\mathscr C^1$ par morceaux sur $\R$. Donc par le théorème de \textsc{Dirichlet}, la série de \textsc{Fourier} de $\varphi$ converge en tout point $x$ de $\R$ :
    \[ \frac 12\left( \varphi(x^+)+\varphi(x^-) \right) = \sum_{n\in\Z}c_n(\varphi)\e^{\i nx} = \left( \e^{\frac z2}-\e^{-\frac z2} \right)\sum_{n\in\Z}\frac{(-1)^n\e^{\i nx}}{z-2\pi\i n}\text. \]
    Appliquons ce résultat à $x=\pi$ pour faire apparaître $f$ : $\dfrac 12\left( \varphi(\pi^+)+\varphi(\pi^-) \right) = \dfrac 12\left( \e^{\frac z2}+\e^{-\frac z2} \right)$, et donc
    \[ \frac 12\left( \e^{\frac z2}+\e^{-\frac z2} \right) = \left( \e^{\frac z2}-\e^{-\frac z2} \right)\sum_{n\in\Z}\frac{(-1)^{2n}}{z-2\i\pi n} = \left( \e^{\frac z2}-\e^{-\frac z2} \right)\sum_{n\in\Z}\frac{z+2\i\pi n}{z^2+4\pi^2n^2}\text. \]
    Or, on peut regrouper les termes de la série en $n$ et $-n$ :
    \[ \sum_{n\in\Z}\frac{z+2\i\pi n}{z^2+4\pi^2n^2} = \frac 1z + \sum_{n=1}^{+\infty}\frac{z+2\i\pi n}{z^2+4\pi^2n^2} + \sum_{n=1}^{+\infty}\frac{z-2\i\pi n}{z^2+4\pi^2n^2} = \frac 1z + 2\sum_{k=1}^{+\infty}\frac z{z^2+4\pi^2n^2}\text. \]
    Il vient donc :
    \[ \frac{\e^{\frac z2}+\e^{-\frac z2}}{2\left( \e^{\frac z2}-\e^{-\frac z2} \right)} = \sum_{n\in\Z}\frac{z+2\i\pi n}{z^2+4\pi^2n^2} = \frac 1z + 2\sum_{k=1}^{+\infty}\frac z{z^2+4\pi^2n^2}\text. \]
    D'autre part, on écrit, en factorisant par $\e^{-\frac z2}$,
    \[ \frac{\e^{\frac z2}+\e^{-\frac z2}}{2\left( \e^{\frac z2}-\e^{-\frac z2} \right)} = \frac{\e^z+1}{2(\e^z-1)} = \frac{\e^z-1+2}{2(\e^z-1)} = \frac 12 + \frac 1{\e^z-1}\text. \]
    Donc en multipliant par $z$ et en soustrayant $\dfrac z2$, on obtient finalement :
    \begin{equation*}
        \forall z\in\C\setminus 2\i\pi\Z,\ f(z) = 1 - \frac z2 + 2\sum_{k=1}^{+\infty}\frac {z^2}{z^2+4\pi^2n^2}\text.\label{eq:1}
    \end{equation*}
    Reste à développement cette somme en série entière. Pour cela, on va développer en série entière la quantité $\dfrac{z^2}{z^2+4\pi^2n^2}$. On a, si $\abs z<2\pi$, $\abs{\dfrac z{2\pi n}}<1$ pour tout $n\in\N^*$. D'où
    \begin{align*}
        \frac{z^2}{z^2+4\pi^2n^2} & = \frac{z^2}{4\pi^2n^2}\frac{1}{1+\dfrac{z^2}{4\pi^2n^2}} = \frac{z^2}{4\pi^2n^2}\sum_{k=0}^{+\infty}(-1)^k\frac{z^{2k}}{(4\pi^2n^2)^k} \\
                                  & = \sum_{k=0}^{+\infty}(-1)^k\frac{z^{2(k+1)}}{(4\pi^2n^2)^{k+1}} = \sum_{k=1}^{+\infty}(-1)^{k-1}\frac{z^{2k}}{(2\pi n)^{2k}}\text.
    \end{align*}
    Notons à présent, pour tout $(n,k)\in(\N^*)^2$, $u_{n,k}\defeq(-1)^{k-1}\dfrac{z^{2k}}{(2\pi n)^{2k}}$. On va montrer que la série double $\sum\sum u_{n,k}$ converge. Pour tout $n\in\N^*$, la série $\displaystyle\sum_k\abs{u_{n,k}}$ converge par hypothèse sur $z$ (série géométrique). De plus, on a
    \[ \forall n\in\N^*,\ \sum_{k=1}^{+\infty}\abs{u_{n,k}} = \frac{\abs z^2}{4\pi^2n^2}\sum_{k=0}^{+\infty}\abs{u_{n,k}} = \frac{\abs z^2}{4\pi^2n^2}\frac 1{1-\dfrac{\abs z^2}{4\pi^2n^2}} = \frac{\abs z^2}{4\pi^2n^2-\abs z^2}\text. \]
    Et par les séries de \textsc{Riemann}, la série $\displaystyle\sum_n\frac{\abs z^2}{4\pi^2n^2-\abs z^2}$ converge. Ainsi, $\displaystyle\sum_n\sum_k \abs{u_{n,k}}$ converge, donc par le théorème de \textsc{Fubini-Lebesgue}, on peut échanger l'ordre de sommation, d'où l'on obtient, pour tout $z\in\C^*$ tel que $\abs z<2\pi$ :
    \begin{align*}
        f(z) & = 1 -\frac z2 + 2\sum_{n=1}^{+\infty}\sum_{k=1}^{+\infty}u_{n,k} = 1 -\frac z2 + 2\sum_{k=1}^{+\infty}\sum_{n=1}^{+\infty}u_{n,k} \\
             & = 1 - \frac z2 + 2\sum_{k=1}^{+\infty}\frac{(-1)^{k-1}}{(2\pi)^{2k}}\Big(\sum_{n=1}^{+\infty}\frac 1{n^{2k}}\Big)z^{2k}\text.
    \end{align*}
    On vérifie aisément que $f$ admet un prolongement par continuité en $0$ défini par $f(0)=1$, et que cette égalité est vérifiée pour $z=0$. On obtient donc
    \[ \forall z\in D(0,2\pi),\ f(z) = 1 - \frac z2 + 2\sum_{k=1}^{+\infty}\frac{(-1)^{k-1}}{(2\pi)^{2k}}\Big(\sum_{n=1}^{+\infty}\frac 1{n^{2k}}\Big)z^{2k}\text. \]
\end{demo}

On note $(b_n)_{n\in\N}$, qu'on appelle la suite des \textbf{nombres de \textsc{Bernoulli}}, les réels tels que
\[ \forall z\in D(0,2\pi),\ f(z) = \sum_{n=0}^{+\infty}\frac{b_n}{n!}z^n\text. \]
On a ainsi $b_0=1$, $b_1=-\dfrac 12$ et $b_{2n+1}=0$ pour tout $n\in\N^*$.

\textbf{Application.}
\[ \forall k\in\N^*,\ \sum_{n=1}^{+\infty}\frac 1{n^{2k}} = (-1)^{k-1}\frac{(2\pi)^{2k}}{2(2n)!}b_{2n}\text. \]

\begin{demo}
    En effet, par unicité du développement en série entière, on a
    \[ \forall k\in\N^*,\ \frac{b_{2k}}{(2k)!} = 2\frac{(-1)^{k-1}}{(2\pi)^{2k}}\sum_{n=1}^{+\infty}\frac 1{n^{2k}}\text, \]
    d'où
    \[ \forall k\in\N^*,\ \sum_{n=1}^{+\infty}\frac 1{n^{2k}} = (-1)^{k-1}\frac{(2\pi)^{2k}}{2(2k)!}b_{2k}\text. \]
\end{demo}

Soit $z\in D(0,2\pi)$. Alors on a, par un produit de \textsc{Cauchy} :
\begin{align*}
    z = f(z)(\e^z-1) & = \Big( \sum_{n=0}^{+\infty}\frac{b_n}{n!}z^n \Big)\Big(\sum_{k=1}^{+\infty}\frac{z^k}{n!}\Big) \\
                     & = \sum_{n=0}^{+\infty}\sum_{k=0}^{n-1}\frac{b_k}{k!}z^k\frac{z^{n-k}}{(n-k)!}                   \\
                     & = \sum_{n=0}^{+\infty}\sum_{k=0}^{n-1}\frac{b_k}{k!(n-k)!}z^n\text.
\end{align*}
On a donc, pour tout $n\in\N\setminus\{0,1\}$,
\[ \sum_{k=0}^{n-1}\frac{b_k}{k!(n-k)!} = 0\text, \]
ce qui veut dire, en multipliant par $n!$,
\[ \sum_{k=0}^{n-1}\binom nk b_k = 0\text. \]
En réarrangeant, on obtient
\[ b_{n-1} = -\frac 1{\binom n{n-1}}\sum_{k=2}^n\binom nk b_k =  -\frac 1{n}\sum_{k=2}^n\binom nk b_k\text. \]
On a donc une relation de récurrence pour calculer les $b_n$. En particulier, on trouve que $b_2 = \dfrac 16$, $b_4=-\dfrac 1{30}$ et $b_6=\dfrac 1{42}$. D'où
\[ \sum_{n=1}^{+\infty}\frac 1{n^{2}} = \frac{\pi^2}6,\quad\sum_{n=1}^{+\infty}\frac 1{n^{4}} = \frac{\pi^4}{90},\quad\sum_{n=1}^{+\infty}\frac 1{n^{6}} = \frac{\pi^6}{945}\text.  \]

\printbibliography
\end{document}